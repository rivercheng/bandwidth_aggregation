\documentclass{article}
\usepackage{fullpage}
\usepackage{setspace}
\title{Support Remote Video Surveillance by Aggregating Multiple 3G Wireless Channels (Draft)}
\author{Cheng Wei\\Center for Maritime Studies\\cmscw@nus.edu.sg}
%\date{}
\begin{document}
\maketitle
\doublespace
\section{Introduction}
    Supporting remote video surveillance on offshore islands
    with undersea fiber-optic cables is expensive. 
    Exploiting the existed 3G wireless networks to support 
    video surveillance is cheaper and easier to maintain.   
    
    Implementing video streaming in delay-sensitive applications
    as video surveillance, however, is challenging due to the characteristics
	of wireless communication. 
    First, in video surveillance systems, cameras send data to the server
    using the up-link (devices to base stations), 
    which usually has less bandwidth than the down-link.
    Moreover, compared with wired links, 
    wireless links have higher delay, jitter, and packet loss rate.
	Having a large buffer in the client side, the common solution to smooth the jitter, 
    increases the end-to-end delay, 
	which is not acceptable in delay-sensitive applications.

	A promising solution is to aggregate multiple 3G wireless channels
    (possibly from multiple service providers) 
    to increase both the bandwidth and quality of service. 
	With multiple channels, the chance of all channels simultaneously suffer 
	from the fading, external interference, or congestion decreases. 
    As a result, the quality of service can be increased by carefully distribute the
    data among the channels based on their real time conditions. 

	Therefore, our study focuses on how to implement a remote video surveillance system
    with satisfying throughput, acceptable delay, and enough reliability 
	by aggregating multiple 3G network channels. 
\section{Related Works}
    Aggregating bandwidth of multiple connections (also called inverse multiplexing)
    has been studied by many researchers recently. 
    These studies mainly target at two problems:
    (i) how to aggregate multiple connections into one virtual connection
    to be used by the applications; and
    (ii) how to properly distribute and schedule data among multiple connections
    to increase reliability and reduce delay and jitter.

	Ideally we should enable the current video applications to use 
	multiple network connections without any modification. 
	For this purpose, usually multiple connections are aggregated 
	into one virtual connection. 
	This aggregation can be implemented in various levels. 
    
    %Aggregation can be implemented at different layers.
	Link layer aggregation is used by 
	ISPs to combine multiple physical links to one logical link 
	(e.g \cite{snoeren:adaptive}). 
	We do not consider implementation on link layer since it involves the
	equipments that only the service provider can control. 
    Moreover, when we use networks provided by multiple service providers, 
    it is not feasible to combine them into one logical link.

    Aggregation in network layer is a common solution used 
    by many researchers \cite{phatak:novel, chebrolu:bandwidth}. Both IP-tunneling and NAT technique
    can be used to provide a virtual IP route transparently to upper layers.
    This method involves the least modification to the system
    and the implementation is easy, but it has several drawbacks.
    First, it is complex to support TCP congestion avoidance scheme
    correctly since the transport layer has no idea that multiple connections exist.
    Even for UDP-based applications, to keep TCP-friendliness on
    each link is difficult if it is required.
    Second, it is hard for the IP layer to do content-aware 
    optimization. Although a TOS field exists in the IP header,
    it is not commonly used by the applications and there is no easy way
    to set them by the applications.

	%Phatak et al. \cite{phatak:ip} proposed a method to aggregate 
	%the channels in network layer with IP-tunneling. 
	%In the sender, a normal IP packet is encapsulated into another
	%IP packet having an extra IP header, 
	%which may include different source IP address. 
	%These IP packets with extra IP header will be sent 
	%through different channels. 
	%At the receiver side the extra header will be removed 
	%in the network layer before delivering to the upper layer. 
	%As a result, the upper layers have no idea multiple connections 
	%are used in the transmission. 

    The most systematic way to support the aggregation is 
    to develop a new protocol in the transport layer. 
	SCTP \cite{rfc4960} uses multiple connections in a 
	single transport layer (virtual) connection, 
	but only one main connection is used to transmit data and others are 
	used for retransmission. 
	El AL et al. Proposed LS-SCTP \cite{elal:transport} as an extension
	to SCTP to use multiple connections simultaneously for data transmission.  
    Support bandwidth aggregation in transport layer is the best 
    way in theory, but in practice it requires modification
    of operation systems and difficult to be accepted widely.
    Moreover, to design and implement a new transport layer protocol
    to be friendly with widely applied TCP is not a easy task.

    Aggregation bandwidth in the application layer \cite{qureshi:tavarua}
    is easy to implement since it involves no modification of operation systems.
    Moreover, applications have the flexibility to control
    the scheduling, so context-aware optimization is easy to be implemented. 
    Modifications, however,  are needed in both the server and the 
    client side. Moreover, how to support the large amount of 
    legacy applications with minimal modification remains a question. 
    Fortunately, in our scenario we have control on both the server and the 
    clients. Moreover, we are developing new application, so aggregation
    bandwidth in the application layer is appropriate.

    The second main problem is how to distribute data and how
    to schedule the transmission over multiple connections
    such that the delay and reordering can be minimized.
    Most of the current studies assume the bandwidth and the
    delay of the channels change slowly and can be easily measured
    on the fly, 
    so the future value can be predicted
    based on the previous average value. 
    Some field experiments, however, show that the bandwidth
    and the delay of wireless channels changes quickly and 
    randomly, which increases the difficulty in developing
    an effective scheduling algorithm.
    Some studies did not consider the packet loss, but 
    bursty packet losses exist in the wireless links and 
    complicate the scheduling problem.

    Channel coding, such as adding FEC packets to protect
    data packets, can be used to reduce the negative effect
    of packet loss and long delay. 
	Cheung et al. proposed FEC distribution algorithms 
	based on their analytical model \cite{cheung:smart}.
    How well their algorithms work under the real 
    3G wireless networks still needs to be verified
    with experiments.

	Being specific to video, 
    Multiple Description Coding (MDC) technique can be used
    to reduce the negative effect of re-ordering caused by 
	sending data through multiple connections, because
    the receiving order of MDC coded packets
    for the same frame does not matter.
    Real time MDC encoding algorithm, however, is computation extensive, 
    and the encoding efficiency is lower than layered encoding
    algorithm. Whether the gain from higher tolerance of reordering
    can compensate the loss of low encoding efficiency
    remains to be examined in experiments.
\section{Research Questions}
    Our research aims to address the above-mentioned gaps and
    implement an prototype of video surveillance system using
    multiple 3G network channels. We need to answer the following
    research questions during the study. 

    \emph{What is the characteristic of 3G network channel and 
    how well it can be measured and predicted?}
    We need to measure what is the real available throughput of
    the 3G networks and how it varies versus time. It is also
    important to do the same measurement on the end-to-end delay.
    With these measured data, we can further analyze how accurate 
    we can predict the future bandwidth and delay based on previous
    average value.
    
    %\emph{Whether transmission data simultaneously on multiple 
    %channels interfere with each other?}
    %We will do experiments for simultaneous transmission on channels
    %provided by the same service provider and on channels provided by
    %different service providers. The answer of this question will
    %significantly affect the design of scheduling algorithm.

    \emph{Whether the channel characteristics are correlated?}
    Similarly, we will check on channels provided by the same
    service provider and on channels provided by different service providers.
    A reasonable hypothesis is that the characteristics of channels
    provided by the same service provider are more correlated than
    the channels provided by different service providers.

    \emph{Could a carefully designed FEC allocation algorithm can
    significantly reduce the negative effect of re-ordering?}
    Because the channel characteristics is variable, 
    packet re-ordering cannot be completely avoided. 
    Using FEC packets can be one way to address this problem.
    Sending FEC packets over multiple connections can not only
    improve the reliability but also reduce the negative effect of long delay,
    because packets suffers from long delay may be recovered when FEC packets
    are received from other connections. As a result, 
    how to adaptively generate and 
    allocate the FEC packets can be an interesting research topic.

    \emph{Is MDC coding can be applicable and helpful?}
    Multiple Description Coding (MDC) can be used to reduce the effect
    of re-ordering and packet loss, because loss of one part of data
    will not affect decoding of other parts of the data. 
    MDC coding, however, decrease the compression efficiency and increase
    the encoding complexity. Whether it is applicable to encode the 
    video on the fly in the camera side and whether it helps to 
    reduce the delay need to be examined. 
\section{Project Approach}
    \subsection{Short-term Approach}
    Currently, our project concentrates on streaming of video surveillance data by using
    the up-links of 3G wireless networks. We use UDP protocol to transmit the video data,
    and optional retransmission of lost packets can be implemented at application level 
    (the retransmitted data may be not useful in the real-time surveillance, but will be
    useful in the recording for future examination).

    To combine the bandwidth from multiple 3G channels, in the application layer, 
    we will implement a demultiplexer in the camera side and a multiplexer in the server side.
    They periodically measure the channel parameters, based on which the demultiplexer
    allocate data and schedule the transmission over multiple channels. FEC packets and MDC
    coding can be used to further reduce the negative effect of packet loss and long delay.

    \subsection{Long-term Approach}
    After we obtain more experience in the short-term approach, we can generalize our system
    to support more channels and more protocols. We plan to include satellite channels and 
    WIMAX channels into consideration so the system can be applied in a wider area. Moreover,
    we will consider streaming using TCP or TCP-friendly UDP protocol to be fair with other 
    TCP flows. Moreover, we will consider bi-direction video and audio transmission
    so that the system can be used in video conference applications.

\section{Project Schedule}
    The proposed research (short-term) will be for four months. 
    
    \begin{itemize}
    \item
    Stage 1 (3 Feb 2010 --  Feb 26 2010) is for literature review and equipment preparation.
    Moreover, we will begin designing the testbed and implement a base prototype. This base prototype
    needs to satisfy two objectives: (i) simultaneously creating UDP connections on two different network 
    interfaces and providing a socket for the applications to use as one virtual connection. 
    (ii) Sending data on the two UDP connections at various data rates and logging the transmission traces
    (including the packet loss and delay).

    \item
    Stage 2 (3 - 4 weeks after equipments are ready) is for field measurement and data analysis. 
    We plan to collect traces for various locations, various moving speeds, and various service providers.
    Then we analyze these data and try to answer the research question 1 and 2. 

    \item
    Stage 3 (4 - 5 weeks after the analysis) is for designing and implementing scheduling algorithms.
    We will do simulations based on the real traces to test our algorithms. FEC allocations will be considered.
    MDC coding can be a possible option.

    \item
    Stage 4 (4 weeks) is for paper writing and prototype implementation. If we obtain satisfying results
    in stage 3, we will write a paper and submit to proper journal or conference. Moreover, we will further 
    develop a usable prototype of streaming system for video surveillance data and test it with field experiments.
    \end{itemize}
    
\section{Project Deliverables}
    \begin{itemize}
    \item
    Publication: we plan to publish a paper to detail the insights and results we obtained from this project.
    \item
    Software: we plan to develop a concept-prove prototype of streaming system for video-surveillance data.
    \end{itemize}
\section{Significance}
    Our algorithm, if shown to be effective, would be used in expanding port surveillance system 
    to remote islands with acceptable costs.

\bibliographystyle{abbrv}
\bibliography{report1}
\end{document}
